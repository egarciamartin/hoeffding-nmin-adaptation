\section{vfkm File Reference}
\label{vfkm}\index{vfkm@{vfkm}}


\subsection{Detailed Description}
Performs k-means clustering accelerated with sampling. 

Performs k-mean clustering accelerated with sampling as described in {\tt this paper}. This learner ignores categorical attributes. vfkm performs several iterations of clustering on progressively larger samples until it determines with high confidence (see -delta below) that the clustering it achieves is within -epsilon of the one that would be achieved using infinite data for each decision. vfkm can use a fancy optimization to select the number of samples to use in each iteration of the next round, or it can use straight progressive sampling. You can use the -batch argument to turn off the sampling and do traditional k-means clustering. vfkm evaluates the learned centers by comparing to the centers found in $<$stem$>$.test as follows. Learned centers are greedily matched to the closest of the test centers until each center has a match, and then the evaluation is the sum of the squared distance between each test center and its matched learned center.

The learner takes input and does output in {\tt c4.5 format}. It expects to find the files {\tt $<$stem$>$.names} and {\tt $<$stem$>$.data.} and outputs the learned centers to a file called {\tt $<$stem$>$.centers}.

\subsubsection*{Arguments}

\begin{itemize}
\item -f 'filestem'\begin{itemize}
\item Set the name of the dataset (default DF)\end{itemize}
\item -source 'dir'\begin{itemize}
\item Set the source data directory (default '.')\end{itemize}
\item -clusters 'num\-Clusters'\begin{itemize}
\item Set the num clusters to find (REQUIRED)\end{itemize}
\item -db\-Size 'examples'\begin{itemize}
\item How many examples are in the datafile (REQUIRED)\end{itemize}
\item -assign\-Error\-Scale 'scale'\begin{itemize}
\item Loosen bound by scaling the assignment part of it by the scale factor (default 1.0)\end{itemize}
\item -delta 'delta'\begin{itemize}
\item Find final solution with confidence (1 - 'delta') (default 5\%)\end{itemize}
\item -epsilon 'epsilon'\begin{itemize}
\item The maximum distance between a learned centroid and the coresponding infinite data centroid (default .05)\end{itemize}
\item -converge 'distance'\begin{itemize}
\item Stop when distance centroids move between iterations squared is less than 'distance' (default .001)\end{itemize}
\item -batch\begin{itemize}
\item Do a traditional kmeans run on the whole input file. Doesn't make sense to combine with -stdin (default off)\end{itemize}
\item -init 'num'\begin{itemize}
\item Use the 'num'th valid assignment of initial centroids (default 1)\end{itemize}
\item -max\-Per\-Iteration 'num'\begin{itemize}
\item Use no more than num examples per iteration (default 1,000,000,000) -l 'number'\item The estimated \# of iterations to converge if you guess wrong and aren't using batch VFKM will fix it for you and do an extra round (default 10)\end{itemize}
\item -seed 's'\begin{itemize}
\item Seed to use picking initial centers (default random)\end{itemize}
\item -progressive\begin{itemize}
\item Don't use our fancy Lagrange based optimization pick the \# of samples at each iteration of the next round (default use the optimization)\end{itemize}
\item -allow\-Bad\-Converge\begin{itemize}
\item Allows VFKM to converge at a time when km would not (default off).\end{itemize}
\item -normal\-Approx\begin{itemize}
\item Use a normal approximation instead of the hoeffding bound (default hoeffding).\end{itemize}
\item -r 'range'\begin{itemize}
\item The maximum range between pairs of examples (default assume the range of each dimension is 0 - 1.0 and calculate it from that)\end{itemize}
\item -stdin\begin{itemize}
\item Reads training examples as a stream from stdin instead of from 'stem'.data (default off)\end{itemize}
\item -test\-On\-Train\begin{itemize}
\item Loss is the sum of the square of the distances from all training examples to their assigned centroid (default compare learned centroids to centroids in 'stem'.test)\end{itemize}
\item -load\-Centers\begin{itemize}
\item Load initial centroids from 'stem'.centers (default use random based on -init and -seed)\end{itemize}
\item -u\begin{itemize}
\item Output results after each iteration\end{itemize}
\item -v\begin{itemize}
\item Can be used multiple times to increase the debugging output\end{itemize}
\item -h\begin{itemize}
\item Run vfkm -h for a list of the arguments and their meanings.\end{itemize}
\end{itemize}


