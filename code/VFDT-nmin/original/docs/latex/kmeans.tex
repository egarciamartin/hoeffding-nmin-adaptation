\section{kmeans File Reference}
\label{kmeans}\index{kmeans@{kmeans}}


\subsection{Detailed Description}
Performs k-means clustering. 

Performs k-mean clustering on the continuous attributes in a data set (ignoring any discrete attributes).

The learner takes input and does output in {\tt c4.5 format}. It expects to find the files {\tt $<$stem$>$.names} and {\tt $<$stem$>$.data.} and outputs the learned centers to a file called {\tt $<$stem$>$.centers}.

Evaluates the learned centers by comparing to the centers found in $<$stem$>$.test as follows. Learned centers are greedily matched to the closest of the test centers until each center has a match, and then the evaluation is the sum of the squared distance between each test center and its matched learned center.

You can find a more full-featured kmeans clustering algorithm by running {\bf vfkm} with the -batch argument (for example you can set initial centroid locations, etc.

\begin{Desc}
\item[{\bf Wish List}]Modify this learner to work with discrete attributes. 

Move the features from {\bf vfkm} into this learner because this learner will be much easier to modify than that one for new users. \end{Desc}


\subsubsection*{Arguments}

\begin{itemize}
\item -f $<$filestem$>$\begin{itemize}
\item Set the stem name (default DF)\end{itemize}
\item -source $<$dir$>$\begin{itemize}
\item Set the directory that contains the dataset (default '.')\end{itemize}
\item -clusters $<$dir$>$\begin{itemize}
\item Sets the number of clusters to look for, this argument is required.\end{itemize}
\item -threshold $<$threshold$>$\begin{itemize}
\item Iterate until every centroid moves less than this threshold.\end{itemize}
\item -u\begin{itemize}
\item Test by comparing to the centroids in $<$threshold$>$.test\end{itemize}
\item -v\begin{itemize}
\item Can be used multiple times to increase the debugging output\end{itemize}
\end{itemize}


