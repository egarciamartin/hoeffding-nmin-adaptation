\section{Hash\-Table.h File Reference}
\label{HashTable_8h}\index{HashTable.h@{HashTable.h}}


\subsection{Detailed Description}
A hash table. 

\begin{Desc}
\item[{\bf Thanks}]to Chun-Hsiang Hung for implementing the {\bf Hash\-Table} ADT. \end{Desc}


\begin{Desc}
\item[{\bf Wish List}]A cleaner hash table interface (without the compare function). I think that the sprint learner is the only thing that uses this currently so there isn't much to change to fix this. \end{Desc}


\subsection*{Data Structures}
\begin{CompactItemize}
\item 
struct {\bf Hash\-Table}
\begin{CompactList}\small\item\em A hash table ADT. \item\end{CompactList}\end{CompactItemize}
\subsection*{Functions}
\begin{CompactItemize}
\item 
{\bf Hash\-Table} $\ast$ {\bf Hash\-Table\-New} (int size)
\begin{CompactList}\small\item\em Creates a new hash table with the specified number of entries. \item\end{CompactList}\item 
void {\bf Hash\-Table\-Insert} ({\bf Hash\-Table} $\ast$table, int index, void $\ast$element)
\begin{CompactList}\small\item\em Inserts the element into the has table at the appropriate place. \item\end{CompactList}\item 
void $\ast$ {\bf Hash\-Table\-Find} ({\bf Hash\-Table} $\ast$table, int index, int($\ast$cmp)(const void $\ast$, const int))
\begin{CompactList}\small\item\em Looks up index in the hash table. \item\end{CompactList}\item 
void {\bf Hash\-Table\-Free} ({\bf Hash\-Table} $\ast$table)
\begin{CompactList}\small\item\em Frees the memory being used by the hash table. \item\end{CompactList}\item 
void $\ast$ {\bf Hash\-Table\-Remove} ({\bf Hash\-Table} $\ast$table, int index, int($\ast$cmp)(const void $\ast$, const int))
\begin{CompactList}\small\item\em Removes the element from the hash table. \item\end{CompactList}\end{CompactItemize}


\subsection{Function Documentation}
\index{HashTable.h@{Hash\-Table.h}!HashTableFind@{HashTableFind}}
\index{HashTableFind@{HashTableFind}!HashTable.h@{Hash\-Table.h}}
\subsubsection{\setlength{\rightskip}{0pt plus 5cm}void$\ast$ Hash\-Table\-Find ({\bf Hash\-Table} $\ast$ {\em table}, int {\em index}, int($\ast$ {\em cmp})(const void $\ast$, const int))}\label{HashTable_8h_a3}


Looks up index in the hash table. 

Not only must you use the same index, but you must also supply a function that returns non-zero when passed the element you want to find and the index (which you passed to the function in the first place...) \index{HashTable.h@{Hash\-Table.h}!HashTableFree@{HashTableFree}}
\index{HashTableFree@{HashTableFree}!HashTable.h@{Hash\-Table.h}}
\subsubsection{\setlength{\rightskip}{0pt plus 5cm}void Hash\-Table\-Free ({\bf Hash\-Table} $\ast$ {\em table})}\label{HashTable_8h_a4}


Frees the memory being used by the hash table. 

But doesn't touch the memory being used by the elements in the table. It is your responsibility to free these if you like. \index{HashTable.h@{Hash\-Table.h}!HashTableInsert@{HashTableInsert}}
\index{HashTableInsert@{HashTableInsert}!HashTable.h@{Hash\-Table.h}}
\subsubsection{\setlength{\rightskip}{0pt plus 5cm}void Hash\-Table\-Insert ({\bf Hash\-Table} $\ast$ {\em table}, int {\em index}, void $\ast$ {\em element})}\label{HashTable_8h_a2}


Inserts the element into the has table at the appropriate place. 

\index{HashTable.h@{Hash\-Table.h}!HashTableNew@{HashTableNew}}
\index{HashTableNew@{HashTableNew}!HashTable.h@{Hash\-Table.h}}
\subsubsection{\setlength{\rightskip}{0pt plus 5cm}{\bf Hash\-Table}$\ast$ Hash\-Table\-New (int {\em size})}\label{HashTable_8h_a1}


Creates a new hash table with the specified number of entries. 

\index{HashTable.h@{Hash\-Table.h}!HashTableRemove@{HashTableRemove}}
\index{HashTableRemove@{HashTableRemove}!HashTable.h@{Hash\-Table.h}}
\subsubsection{\setlength{\rightskip}{0pt plus 5cm}void$\ast$ Hash\-Table\-Remove ({\bf Hash\-Table} $\ast$ {\em table}, int {\em index}, int($\ast$ {\em cmp})(const void $\ast$, const int))}\label{HashTable_8h_a5}


Removes the element from the hash table. 

Not only must you use the same index, but you must also supply a function that returns non-zero when passed the element you want to find and the index (which you passed to the function in the first place...) 